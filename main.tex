\documentclass{article}
\usepackage{graphicx}
\usepackage[margin=2.5cm]{geometry}
\usepackage{float}
\usepackage{amsmath}
\usepackage{parskip}


\begin{document}
\title{Assignment 8}
\author{Cherryl Chico, Mariajosé Argote}
\date{2 December 2025}
\maketitle

\section*{Exercise 1.1}
We replicate Figure 1 from Card and Krueger (1994) using the provided data. The variable \texttt{wage\_st} is the state minimum wage, \texttt{time} indicates the survey wave (0: Feb 1992, 1: Nov 1992), and \texttt{state} distinguishes New Jersey (1) from Pennsylvania (0). We bin \texttt{wage\_st} in 0.1 dollar intervals and compute the percentage of stores per bin for each state and wave. The resulting plot shows the distribution of \texttt{wage\_st} by state and wave, with the x-axis as bin averages and the y-axis as store percentages.

\begin{figure}[ht]
    \centering
    \includegraphics[height=0.5\textheight]{Figure1.png}
    \caption{Distribution of Starting Wage Rates}
    \label{fig:figure1}
\end{figure}

The figure replicates the original findings, showing a shift in New Jersey's wage distribution after the minimum wage increase, while Pennsylvania's distribution remains generally unchanged.

\section*{Exercise 1.2}
We estimate the following model, following Card and Krueger (1994):
\begin{equation*}
    \Delta E_i = \alpha_0 + \boldsymbol{\beta} \mathbf{X}_i + \gamma\, \text{NJ}_i + \epsilon_{i}
\end{equation*}
Here, $\Delta E_i$ represents the change in employment (FTE) at store $i$, $\mathbf{X}_i$ includes store characteristics, and $\text{NJ}_i$ is a New Jersey dummy.

We restrict the sample to stores with complete employment and wage data in both waves (351 stores). We calculate FTE as the sum of full-time workers, half the number of part-time workers, and number of managers. For each store, we compute the change in FTE between waves. To check robustness, we run two regressions: (1) using only the state dummy, and (2) adding controls for chain affiliation and co-ownership. We report the coefficient for the state dummy in both models.

We also estimate the model using a difference-in-differences (DiD) approach with the full panel data:
\begin{equation*}
    E_{it} = \alpha_0 + \alpha_1\, \text{NJ}_i + \alpha_2\, \text{Post}_t + \gamma\, (\text{NJ}_i \times \text{Post}_t) + \epsilon_{it}
\end{equation*}
In this specification, $E_{it}$ is employment at store $i$ in time $t$, $\text{Post}_t$ is a post-treatment time dummy, and the interaction term captures the treatment effect. We calculate the change in employment per store and run the OLS regressions as described above, then compare the results of both approaches.

In total, we estimate four models: two following the original paper and two using the DiD approach. Table \ref{tab:regression_results} reports the NJ dummy coefficient, standard error, and p-value for each model.

\begin{table}[H]
    \centering
    \begin{tabular}{lccccc}
        \hline
        Model & Variable & Coefficient & Std. Error & p-value \\
        \hline
        OLS without controls & NJ & 2.28 & 1.19 & 0.06 \\
        OLS with controls & NJ&  2.28 & 1.20 & 0.06 \\
        DiD without controls & NJ:Post & 2.28 & 1.80 & 0.21 \\
        DiD with controls & NJ:Post & 2.28 & 1.60 & 0.16 \\
        \hline
    \end{tabular}
    \caption{Regression Results for Change in Employment}
    \label{tab:regression_results}
\end{table}

The fact that the point estimate is stable across specifications suggests that the result is not driven by the inclusion of controls or by the particular way we parameterise the model (change scores vs. DiD). What changes is only the precision.

We interpret these results as providing no evidence that the 1992 NJ minimum wage increase reduced employment in fast-food restaurants. If anything, the point estimates point to a small positive effect on FTE employment relative to PA. From a causal inference perspective, under the parallel-trends assumption this is inconsistent with a simple competitive labor market story and more in line with models where firms have some wage-setting power or adjust on other margins rather than cutting jobs.

\section*{Exercise 1.3}
We examine whether the policy affected full meal prices, following Card and Krueger (1994):

\begin{equation*}
    \Delta P_i = \alpha_0 + \boldsymbol{\beta} \mathbf{X}_i + \gamma\, \text{NJ}_i + \epsilon_{i}
\end{equation*}

$P_i$ is the log price of a full meal (soda + fries + entree). We compute the change in log price per store, restricting to those with complete employment and price data in both waves in addition to the filters from Exercise 1.2 (314 stores). Two OLS regressions are run: (1) NJ dummy only, (2) with chain and co-ownership controls. We also estimate the model using a DiD approach with the full panel.

\begin{table}[H]
    \centering
    \begin{tabular}{lccccc}
        \hline
        Model & Variable & Coefficient & Std. Error & p-value \\
        \hline
        OLS without controls & NJ & 0.033 & 0.014 & 0.025\\
        OLS with controls & NJ&  0.037 & 0.014 & 0.009 \\
        DiD without controls & NJ:Post & 0.033 & 0.036 & 0.368 \\
        DiD with controls & NJ:Post & 0.033 & 0.019 & 0.094 \\
        \hline
    \end{tabular}
    \caption{Regression Results for Change in Log of Full Meal Price}
    \label{tab:regression_results}
\end{table}

Because the dependent variable is a change in log prices, $\hat{\gamma}$ can be interpreted as the difference in percentage price changes between NJ and PA. The point estimate of $0.037$ implies that, on average, full-meal prices in New Jersey increased by about $0.033 \text{ to } 0.037$ percent more than in PA over the period.

Under the parallel trends assumption, we interpret $\hat{\gamma}$ as the causal effect of the minimum wage on menu prices. The positive and statistically significant estimate suggests that New Jersey restaurants passed on part of the higher labor costs to consumers in the form of higher prices.

This pattern is consistent with the presence of product market power. If fast-food restaurants faced perfectly competitive product markets, we would expect limited scope to raise prices relative to nearby Pennsylvania competitors without losing demand. Instead, we find that NJ fast food restaurants were able to increase prices modestly but significantly relative to PA. The magnitude of the effect is not huge in levels, but it is also not zero. Considering this and the employment results, it supports the idea that some of the adjustment to the minimum wage increase was transferred through prices, in a setting where firms possess at least some ability to mark up prices above marginal cost.


\end{document}